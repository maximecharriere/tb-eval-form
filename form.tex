\documentclass[a4paper]{article}
\usepackage[T1]{fontenc}
\usepackage[utf8]{inputenc}
\usepackage{geometry}
\geometry{textwidth=17cm,top=1cm,bottom=0.5cm}
\usepackage{hyperref}[2012/10/12]
\usepackage{tabularx,booktabs}
\usepackage{graphicx}
\usepackage[table]{xcolor}
\usepackage{tikz}
\usepackage{multirow}
\usepackage{sectsty}
\usepackage{digsig}
\setlength\parindent{0pt}
\allsectionsfont{\normalfont\sffamily\bfseries}

\begin{document}
\thispagestyle{empty}
\begin{tikzpicture}[remember picture,overlay]
    \node[anchor=north west,yshift=-2cm,xshift=2cm]%
        at (current page.north west)
        {\includegraphics[height=2cm]{assets/logos/heig-vd-baseline.pdf}};
    \end{tikzpicture}
\fontfamily{lmss}\selectfont
\vspace{3cm}
\begin{center}
{\usefont{T1}{lmss}{bx}{n}\huge\selectfont
Évaluation intermédiaire\\\vspace{0.5em}
\large{projet multidisciplinaire}\par}%
\end{center}

\begin{Form}
 
\begin{tabularx}{\textwidth}{|l|X|l|l|} \hline
    \textbf{Professeur} & Yves Chevallier & \textbf{Date} & 2021-05-09 \\ \hline
    \textbf{Titre du projet} & \multicolumn{3}{l|}{Les chocolats bleus} \\ \hline
    \multirow{2}{*}{\textbf{Étudiants}} & \textbf{Nom} & \textbf{Prénom} & \textbf{Filière} \\
    & Chevallier & Yves & EAI \\ [2cm] \hline


\end{tabularx}
\vskip 2em

\newcommand\weight[2]{%
    \TextField[
        maxlen=2,align=1,height=10pt,width=1.5cm,name=#1,default=#2,
        format={AFNumber_Format\string\(0, 2, 0, 0, "\%", false \string\);}]{}
}

\newcommand\score[1]{%
    \TextField[
        maxlen=3,align=1,height=10pt,width=1.5cm,name=#1,default=1.0,
        calculate={%
            event.value = Math.min(6.0, Math.max(1.0, this.getField('#1').value))
        },
        format={AFNumber_Format\string\(1, 2, 0, 0, "", false \string\);}]{}
}

\newcommand\porg{%
    \TextField[maxlen=2,align=1,height=10pt,width=1.5cm,name=p1,
  calculate={%
    event.value =
        this.getField("p2").value +
        this.getField("p3").value +
        this.getField("p4").value +
        this.getField("p5").value +
        this.getField("p6").value
  },readonly=true,format={AFNumber_Format\string\(0, 2, 0, 0, "\%", false \string\);}]{}
}

\newcommand\sorg{%
    \TextField[maxlen=2,align=1,height=10pt,width=1.5cm,name=n1,
  calculate={%
    event.value = (
        this.getField("p2").value * this.getField("n2").value +
        this.getField("p3").value * this.getField("n3").value +
        this.getField("p4").value * this.getField("n4").value +
        this.getField("p5").value * this.getField("n5").value +
        this.getField("p6").value * this.getField("n6").value)
        / this.getField("p1").value
  }, readonly=true,
  format={AFNumber_Format\string\(1, 2, 0, 0, "", false \string\);}]{}
}

\newcommand\pcom{%
    \TextField[maxlen=2,align=1,height=10pt,width=1.5cm,name=p7,
  calculate={%
    event.value =
        this.getField("p8").value +
        this.getField("p9").value
  },readonly=true,
  format={AFNumber_Format\string\(0, 2, 0, 0, "\%", false \string\);}]{}
}

\newcommand\scom{%
    \TextField[maxlen=2,align=1,height=10pt,width=1.5cm,name=n7,
  calculate={%
    event.value = (
        this.getField("p8").value * this.getField("n8").value +
        this.getField("p9").value * this.getField("n9").value)
        / this.getField("p7").value
  }, readonly=true,
  format={AFNumber_Format\string\(1, 2, 0, 0, "", false \string\);}]{}
}




\newcommand\hs{\hspace{2em}}
\newcommand\tr{\rowcolor{lightgray}}
\renewcommand\tabularxcolumn[1]{m{#1}}

\begin{tabularx}{\textwidth}{|X|c|c|} \hline
    Critère d'évaluation & Pondération & Note \\\hline

    \tr Organisation du projet                   & \porg            & \sorg \\[3ex] \hline
    \hs Précision de la spécification            & \weight{p2}{10}  & \score{n2} \\
    \hs Validation de la spécification           & \weight{p3}{10}  & \score{n3} \\
    \hs Planification du travail                 & \weight{p4}{10}  & \score{n4} \\
    \hs Distribution efficace du travail         & \weight{p5}{10}  & \score{n5} \\
    \hs Suivi régulier du projet                 & \weight{p6}{10}  & \score{n6} \\ \hline

    \tr Communication                            & \pcom  & \scom \\[3ex]
    \hs Consultation régulière du professeur
        répondant et des experts                 & \weight{p8}{10}  & \score{n8} \\
    \hs Implication et collaboration de l'équipe & \weight{p9}{10}  & \score{n9} \\ \hline

    \tr Technique                                & \weight{p10}{10} & \score{n10} \\[3ex]
    \hs Adéquation et aboutissement du concept   & \weight{p11}{10} & \score{n11} \\
    \hs Pertinence des choix technologiques      & \weight{p12}{10} & \score{n12} \\ \hline

    \tr Avancement                               & \weight{p13}{10} & \score{n13} \\[3ex]
    \hs Respect du planning                      & \weight{p14}{10} & \score{n14} \\

    \rowcolor{lightgray} Évaluation globale      & \weight{p15}{10} & \score{n15} \\[3ex] \hline
\end{tabularx}

\newcommand\clearrow{\global\let\rowmac\relax}
\newcommand\setrow[1]{\gdef\rowmac{#1}#1\ignorespaces}
\newcolumntype{Y}{>{\centering\arraybackslash\rowmac}X}
%
\vskip 2em
\section*{Échelle d'évaluation}
\clearrow
\begin{tabularx}{\textwidth}{|*{6}{Y|}Y<{\clearrow}|} \hline
\setrow{\bfseries} Excellent & Très bien & Bien & Satisfaisant & Passable & Échec & Échec \\ \hline
A & B & C & D & E & FX & F \\ \hline
5.8 à 6.0 & 5.3 à 5.7 & 4.8 à 5.2 & 4.3 à 4.7 & 4.0 à 4.2 & 3.5 à 3.9 & 1.0 à 3.4 \\ \hline
\end{tabularx}


\begin{tabularx}{\textwidth}{p{12cm}X}
    \section*{Remarques} &  \\
\noindent\fbox{%
\parbox{12cm}{%
\TextField[maxlen=2,align=1,height=2cm,width=12cm,name=comment]{}

}} & \digsigfield{5cm}{2cm}{} \\
\end{tabularx}

\end{Form}
\end{document}
